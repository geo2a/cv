%%%%%%%%%%%%%%%%%%%%%%%%%%%%%%%%%%%%%%%%%
% "ModernCV" CV and Cover Letter
% LaTeX Template
% Version 1.3 (29/10/16)
%
% This template has been downloaded from:
% http://www.LaTeXTemplates.com
%
% Original author:
% Xavier Danaux (xdanaux@gmail.com) with modifications by:
% Vel (vel@latextemplates.com)
%
% License:
% CC BY-NC-SA 3.0 (http://creativecommons.org/licenses/by-nc-sa/3.0/)
%
% Important note:
% This template requires the moderncv.cls and .sty files to be in the same
% directory as this .tex file. These files provide the resume style and themes
% used for structuring the document.
%
%%%%%%%%%%%%%%%%%%%%%%%%%%%%%%%%%%%%%%%%%

%----------------------------------------------------------------------------------------
%	PACKAGES AND OTHER DOCUMENT CONFIGURATIONS
%----------------------------------------------------------------------------------------

\documentclass[12pt,a4paper,sans,colorlinks,urlcolor=cyan,linkcolor=blue]{moderncv} % Font sizes: 10, 11, or 12; paper sizes: a4paper, letterpaper, a5paper, legalpaper, executivepaper or landscape; font families: sans or roman

% Underline my name in publications
\newcommand{\myname}[1]{\underline{Georgy Lukyanov}}

% https://tex.stackexchange.com/questions/310612/moderncv-date-year-on-top-month
\usepackage{datetime}
\def\cvdate[#1.#2]{\makebox[25mm]{\footnotesize%
    \parbox[t]{10mm}{\centering{\shortmonthname[#1]\par%
        {#2}}}}%
  }
\def\cvyears[#1-#2]{\makebox[25mm]{\footnotesize%
    \parbox[t]{12mm}{\centering{{#1}}}%
    \makebox[5mm]{{--}}%
    \parbox[t]{12mm}{\centering{{#2}}}%
}}
\def\cvdates[#1.#2-#3.#4]{\makebox[25mm]{\footnotesize%
    \parbox[t]{10mm}{\centering{\shortmonthname[#1]\par%
        {#2}}}%
    \makebox[5mm]{\raisebox{-0.5ex}{{--}}}%
    \parbox[t]{10mm}{\centering{\shortmonthname[#3]\par%
        {#4}}}
}}

\usepackage [maxbibnames=10, sorting=none, style = numeric] {biblatex}
\defbibheading{bibliography}[\refname]{}
\addbibresource {publications.bib}

\moderncvstyle{classic} % CV theme - options include: 'casual' (default), 'classic', 'oldstyle' and 'banking'
\moderncvcolor{blue} % CV color - options include: 'blue' (default), 'orange', 'green', 'red', 'purple', 'grey' and 'black'
\usepackage{paratype}
% \usepackage[scale=0.9]{geometry} % Reduce document margins
\usepackage{geometry}
 \geometry{
 a4paper,
 total={170mm,257mm},
 left=15mm,
 right=15mm,
 top=10mm,
 bottom=10mm
}

\setlength{\hintscolumnwidth}{2.6cm} % Uncomment to change the width of the dates column
%\setlength{\makecvtitlenamewidth}{10cm} % For the 'classic' style, uncomment to adjust the width of the space allocated to your name

%----------------------------------------------------------------------------------------
%	NAME AND CONTACT INFORMATION SECTION
%----------------------------------------------------------------------------------------

\firstname{Georgy} % Your first name
\familyname{Lukyanov} % Your last name

% All information in this block is optional, comment out any lines you don't need
\title{Curriculum Vitae}
\email{mail@geo2a.info}
\homepage{geo2a.info}{geo2a.info} % The first argument is the url for the clickable link, the second argument is the url displayed in the template - this allows special characters to be displayed such as the tilde in this example
% \extrainfo{additional information}
% \photo[70pt][0.4pt]{pictures/picture} % The first bracket is the picture height, the second is the thickness of the frame around the picture (0pt for no frame)
% \quote{"A witty and playful quotation" - John Smith}

% Colored hyperlinks
% \newcommand\Colorhref[3][cyan]{\href{#2}{\small\color{#1}#3}}

%----------------------------------------------------------------------------------------

\begin{document}
%----------------------------------------------------------------------------------------
%	CURRICULUM VITAE
%----------------------------------------------------------------------------------------

\makecvtitle % Print the CV title
%-------------------------------------------------------------------------------
% RESEARCH INTERESTS SECTION
%-------------------------------------------------------------------------------

\vspace{-10mm}

I am a final stage PhD student at Newcastle University working on functional programming and computer engineering. The primary subject of my PhD is the development of a symbolic execution-based verification framework for a specialised instruction-set architecture for spacecraft subsystems. I research how functional programming techniques can help to structure the implementation of such a framework. 

\setlength\parindent{0pt} 
%----------------------------------------------------------------------------------------
% ACADEMIC PUBLICATIONS SECTION
%----------------------------------------------------------------------------------------

\vspace{-3mm}
\section{Selected Publications}

\nocite{*}
\printbibliography

%----------------------------------------------------------------------------------------
%	WORK EXPERIENCE SECTION
%----------------------------------------------------------------------------------------

\vspace{-7mm}
\section{Research Visits}

\vspace{-2mm}
\cventry{\cvdates[02.2020-04.2020]}{Vrije Universiteit Brussel}{}{}{}{I worked with Dominique Devriese and Steven Keuchel on \href{https://github.com/skeuchel/katamaran}{Katamaran} --- a Coq-embedded subset of the \href{https://www.cl.cam.ac.uk/~pes20/sail}{SAIL} instruction-set architecture specification language. My contributions were a generic separation logic interface and a disjoint-heap model for it, complete with a soundness proof with respect to SAIL's operational semantics. }

\cventry{\cvdate[04.2019]}{TU Wien and RUAG Space Austria}{}{}{}{I visited for two weeks to present updates on the verification framework for the REDFIN processor and work on new verification examples.}

\cventry{\cvdate[07.2018]}{Cambridge University Computer Laboratory}{}{}{}{I spent a week there to meet researchers from Programming, Logic, and Semantics and Computer Architecture groups and get a closer look at the \href{https://www.cl.cam.ac.uk/~pes20/sail}{SAIL} language. I gave a presentation at the department seminar.
}

\vspace{-5mm}
\section{Work Experience}

\vspace{-2mm}
\cventry{\cvdates[10.2017-08.2020]}{Teaching Assistant}{\textsc{Newcastle University}}{}{}{
C programming; Computer Systems and Microprocessors; Real Time and Embedded Systems
}

\cventry{\cvdates[06.2015-06.2017]}{Software Engineer}{\textsc{Statzilla}}{Rostov-on-Don, Russia}{}{
Machine learning, statistics and digital signal processing. Web programming.
}
%----------------------------------------------
\cventry{\cvdates[09.2015-12.2015]}{Teaching Assistant}{\textsc{Southern Federal University}}{Rostov-on-Don, Russia}{}{
Teaching ``Functional Programming in Haskell'' labs with occasional lecturing.
}
%----------------------------------------------------------------------------------------
%   EDUCATION SECTION
%----------------------------------------------------------------------------------------
\vspace{-5mm}
\section{Education}

\cventry{\cvyears[2017-Present]}{PhD Student in Computer Engineering}{Newcastle University}{UK}{}{}
\cventry{\cvyears[2015-2017]}{M.Sc. in Computer Science (with distinction)}{Southern Federal University}{Russia}{}{}  % Arguments not required can be left empty
% \cventry{February --\\June 2016}{Eramus+ exchange program}{Vilnius University, Faculty of Mathematics and Informatics, Faculty of Philology}{Vilnius, Lithuania}{}{}

\cventry{\cvyears[2011-2015]}{B.Sc. in Applied Mathematics}{Southern Federal University}{Russia}{}{}
%-------------------------------------------------------------------------------
% \vspace{-5mm}
% \section{Qualifications}
% \begin{minipage}[t]{.45\textwidth}
% {\large \textbf{Programming languages}} \\
%     % \begin{subsection}
%       Haskell, Coq, R, C, JavaScript.
%     % \end{subsection}
% \end{minipage}
% \hfill
% \noindent
% \begin{minipage}[t]{.45\textwidth}
% {\large \textbf{Spoken languages}} \\
%     % \begin{subsection}
%       \emph{English}: C1, IELST overall score: 8.0. \\
%       \emph{Russian}: native.
%     % \end{subsection}
% \end{minipage}
  

  
\end{document}
